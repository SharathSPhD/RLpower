\section{Conclusion and Future Work}
\label{sec:conclusion}
%% ---------------------------------------------------------------------------

We have demonstrated a complete RL control pipeline for sCO$_2$ waste-heat
recovery that:
\begin{itemize}
  \item Successfully traverses all 7 curriculum phases (Phases 0--6) within
        229{,}376 training steps once infrastructure bugs are resolved, and
        completes a 5{,}013{,}504-step training run with final evaluation
        reward 412.7 and zero constraint violations.
  \item Achieves \textbf{+24--29\% improvement} over PID in Phases~0--2
        (steady-state, gradual load following, ambient disturbance) with
        zero constraint violations across all 70~evaluation episodes.
  \item Reveals a \textbf{curriculum phase imbalance failure mode}: spending
        95\% of training in Phase~6 causes catastrophic forgetting of
        Phases~3--5 skills, a finding with broad applicability to non-interleaved
        curriculum RL.
  \item Satisfies deployment latency requirements with p99 = 0.046~ms
        ($22\times$ margin against the 1~ms SLA), providing a production-ready
        inference artefact.
  \item Provides a \textbf{reusable SCOPE controller library}
        (\texttt{sco2rl.control}) with a common \texttt{Controller} interface
        for PID and RL policies, IMC-tuned \texttt{MultiLoopPID} with
        anti-windup and derivative filter, and an \texttt{sco2rl.analysis}
        module for standardised step-response and frequency-domain benchmarking
        across all curriculum phases.
\end{itemize}

The control-theoretic analysis (Section~\ref{sec:control_analysis}) reveals
that the IMC-tuned PID baseline achieves large stability margins (GM~$\geq$40~dB,
PM~$\geq$45\textdegree{}) on the linearised dynamics but exhibits 66\% overshoot
on the bypass-valve channel due to the asymmetric valve authority of the
sCO$_2$ cycle.  Phase~3 and Phase~6 disturbances produce 4--10$\times$ higher
IAE than Phase~0 for the decoupled PID, underscoring the need for a
multi-channel controller that accounts for the thermodynamic coupling absent in
independent proportional-integral loops.
The RL policy implicitly learns this coupling structure, explaining the
+24--29\% reward improvement over PID in Phases~0--2 — a finding that is now
rigorously quantified through both step-response and frequency-domain metrics
in the \texttt{sco2rl.analysis} library.

The five practitioner bugs documented in Section~\ref{sec:bugs} represent
the most practically useful contribution: they are not algorithmic failures
but engineering failures in the training infrastructure, each with a clear
diagnosis, fix, and detection strategy for future practitioners.

Two further training experiments are in progress or planned:
\begin{enumerate}
  \item \textbf{Interleaved curriculum training (in progress).}
        A supplementary 3M-step run resumes from the 5M checkpoint with
        per-worker phase interleaving (30\% of Phase-6 workers periodically
        replay a random Phase~0--5 episode) implemented via the new
        \texttt{interleave\_ratio} parameter in \texttt{CurriculumCallback}.
        Results will be reported as an addendum; regardless of per-phase
        outcome, both result sets (5M-only and interleaved) will be preserved
        in the final manuscript as they quantify the forgetting effect and
        its remediation.
  \item \textbf{Surrogate dataset collection and retraining.}
        Collect $100{,}000$ strictly unique LHS FMU trajectories with
        critical-region oversampling and retrain FNO; target $R^2 > 0.8$.
        This would enable GPU-vectorised training at ${\sim}10^6$~steps/s,
        reducing wall-clock time by ${\sim}2000\times$.
\end{enumerate}

Interactive Jupyter notebooks demonstrating cycle analysis, reward shaping
diagnostics, surrogate validation, and policy evaluation are provided
alongside the code release, enabling reproducible exploration without an
FMU runtime via Google Colab.

%% ---------------------------------------------------------------------------
