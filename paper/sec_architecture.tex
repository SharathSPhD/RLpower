\section{System Architecture}
%% ---------------------------------------------------------------------------

\begin{figure}[t]
  \centering
  \includegraphics[width=0.90\linewidth]{figures/cycle_diagram.png}
  \caption{%
    Simple recuperated sCO$_2$ Brayton cycle schematic.
    The clockwise flow path connects: heat source (EAF/BOF exhaust,
    200--1200\textdegree{}C) $\to$ turbine [2] $\to$ recuperator hot side [4,5]
    $\to$ pre-cooler [5$\to$1] $\to$ main compressor [1$\to$2] $\to$ recuperator
    cold side [2$\to$3] $\to$ heat source.
    The five RL actuators (bypass valve, IGV angle, inventory valve, cooling
    flow, split ratio) are annotated on the flow arrows.
    Critical safety constraint: compressor inlet must remain above 33\textdegree{}C
    (1.9\textdegree{}C above the CO$_2$ critical point).%
  }
  \label{fig:cycle_schematic}
\end{figure}

\begin{figure}[t]
  \centering
  \includegraphics[width=0.78\linewidth]{figures/cycle_ts_diagram.png}
  \caption{%
    CO$_2$ temperature-entropy (T-s) diagram showing the saturation dome
    (blue), six reference state points at the design operating point
    (P$_{\text{high}}$=20~MPa, P$_{\text{low}}$=7.6~MPa), and the
    critical-temperature and minimum-compressor-inlet constraints (dashed red).
    Near-critical specific heat peaks ($c_p\!\approx\!29.6$~kJ/(kg$\cdot$K)
    at 35\textdegree{}C/80~bar) create the asymmetric nonlinearity that
    motivates RL control.%
  }
  \label{fig:cycle_ts}
\end{figure}

\subsection{Physics Simulation Layer}

The base environment is a \emph{simple recuperated} sCO$_2$ Brayton cycle
(Figures~\ref{fig:cycle_schematic} and~\ref{fig:cycle_ts})
modelled in OpenModelica (OM~1.23) using ThermoPower~\cite{thermopower} and
ExternalMedia~\cite{externalmedia} with the CoolProp Span--Wagner
CO$_2$ EOS~\cite{coolprop}.
The model is exported as an FMI~2.0 Co-Simulation FMU with the CVODE
stiff solver embedded (\texttt{--fmiFlags=s:cvode}, relative tolerance $10^{-4}$).
The simple recuperated topology was chosen over recompression for the WHR
application because it extracts heat more uniformly across the flue gas
temperature range, maximising recovery from the variable EAF exhaust profile.

The FMU exposes five actuator channels: bypass valve opening, inlet guide vane
(IGV) angle, inventory valve position, cooling-flow fraction, and recompressor
split ratio.
Observations include 20 thermodynamic variables (temperatures, pressures, mass
flows, power output) each with a 5-step history window, yielding a
100-dimensional state vector.

Five critical engineering constraints are enforced:
\begin{itemize}
  \item Compressor inlet temperature $T_{\text{ci}} \geq 32.2$\textdegree{}C
        (1.1\textdegree{}C above the critical point).
        Dropping below 31.5\textdegree{}C triggers immediate episode
        termination with reward $-100$.
  \item Surge margin $\sigma \geq 0.05$ to prevent compressor stall.
  \item Turbine inlet temperature within design envelope.
  \item High-side pressure within mechanical limits.
  \item Net power output non-negative (no parasitic consumption).
\end{itemize}

\subsection{Gymnasium Environment}

\texttt{SCO2FMUEnv} wraps the FMU via FMPy (preferred over PyFMI for its
zero-C-extension installation) with an explicit unit-conversion layer
(\texttt{FMPyAdapter.default\_scale\_offset()}) that converts FMU-native
SI units (watts) to engineering units (MW) \emph{before} the reward
function observes them.
Key components:
\begin{itemize}
  \item \textbf{Observation}: 20 variables $\times$ 5 history steps
        $= 100$-dimensional input vector.
  \item \textbf{Action}: 5-dimensional continuous in $[-1,1]$, decoded to
        physical ranges and rate-limited to prevent actuator damage.
  \item \textbf{Normalisation}: SB3 \texttt{VecNormalize} with running
        mean/variance across all 8~parallel environments;
        must be persisted alongside policy weights at every checkpoint
        (see Section~\ref{sec:bugs}).
  \item \textbf{Reward}: $r = r_{\text{tracking}} + r_{\text{smooth}} - r_{\text{constraint}}$,
        where $r_{\text{tracking}}$ rewards net power output towards the
        demand setpoint, $r_{\text{smooth}}$ penalises excessive actuator
        movement, and $r_{\text{constraint}}$ penalises physical limit violations.
\end{itemize}

\subsection{RL Training}

PPO~\cite{schulman2017} is implemented via Stable-Baselines3~\cite{sb3} with
Lagrangian constraint multipliers~\cite{achiam2017cpo} attached as trainable
parameters $\lambda_c \geq 0$ updated online from per-step violation signals.
The actor and critic share an MLP backbone with hidden layers $[256, 256, 128]$
(${\approx}400$K parameters).
Key hyperparameters: clip $\varepsilon = 0.2$, GAE $\lambda_{\text{GAE}} = 0.95$,
$\gamma = 0.99$, learning rate $3\times10^{-4}$ (linear decay),
mini-batch 256, epochs 10, rollout steps $n_{\text{steps}} = 2{,}048$.

Training uses two parallel paths:
\begin{enumerate}
  \item \textbf{FMU path}: SB3 PPO + \texttt{SubprocVecEnv}
        (8 parallel FMU instances on CPU); throughput ${\approx}530$~steps/s.
  \item \textbf{Surrogate path}: SKRL PPO + 1{,}024-way GPU vectorisation
        backed by an FNO surrogate; throughput ${\approx}10^6$~steps/s.
\end{enumerate}

\subsection{Curriculum Learning}

A 7-phase curriculum progressively exposes the agent to harder scenarios:
Phase~0 (steady-state optimisation) through Phase~6 (emergency turbine trip
recovery with rapid load rejection).
Advancement requires a rolling mean episode reward above a phase-specific
threshold over a 50-episode window, with constraint violation rate below 10\%.

\subsection{Surrogate Model}

The FNO~\cite{fno2021} is a 1-D operator learning model that maps
$(s_t, a_t) \mapsto s_{t+1}$ in normalised coordinates.
Architecture: 4~Fourier layers, 64~modes, 128~width, GELU activation.
After GPU-surrogate training, 500K fine-tuning steps on the live FMU
correct surrogate bias.

\subsection{Deployment Path}

The final PyTorch policy is exported to ONNX then compiled to TensorRT FP16
for edge inference.
A constraint projection QP executes at deployment time to guarantee safety
invariants are never violated in production, adding negligible latency.

%% ---------------------------------------------------------------------------
