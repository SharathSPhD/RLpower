\section{Related Work}
%% ---------------------------------------------------------------------------

\subsection{RL for Thermodynamic Cycle Control}

Deep RL has been applied to related thermodynamic control problems.
Wang et al.~\cite{wang2020orc} demonstrated that a soft actor-critic agent
outperforms PID control for ORC superheat regulation under highly transient
internal combustion engine exhaust, achieving superior generalisation to
unseen disturbance profiles.
A 2022 study on sCO$_2$ recompression cycle control~\cite{sco2compare2021}
compared multiple conventional control strategies (PI, feedforward, combined)
during load-following transients, establishing the baseline landscape that RL
must surpass.

For the sCO$_2$ cycle specifically, Zhu et al.~\cite{zhu2024fno} recently
applied Fourier Neural Operators to characterise open-loop transient dynamics
and embed the learned model in a Model Predictive Controller, demonstrating
that the printed circuit heat exchanger outlet temperature exhibits non-minimum
phase characteristics that defeat simple PI feedback — precisely the scenario
our RL curriculum is designed to handle.

\subsection{FMU-Based RL Environments}

ModelicaGym~\cite{modelicagym} provides a Gym wrapper for Modelica FMUs
validated on Cart-Pole.
BOPTEST-Gym~\cite{boptestgym} targets building HVAC systems with FMU
simulation; its benchmarking framework informs our evaluation approach.
FMUGym~\cite{fmugym} and OpenModelica-Microgrid-Gym~\cite{ommicrogrid}
cover electrical power networks.
None of these frameworks address thermodynamic power cycles, curriculum
learning, or Lagrangian safety constraints.

\subsection{Lagrangian Safe RL}

Constrained Policy Optimisation (CPO)~\cite{achiam2017cpo} established
the theoretical foundation for policy search with hard constraints.
Our Lagrangian relaxation approach maintains trainable multipliers
$\lambda_c \geq 0$ updated via gradient ascent on the constraint dual,
providing a practical implementation that does not require trust-region
solves at each step while still converging to constraint satisfaction in
training.

%% ---------------------------------------------------------------------------
