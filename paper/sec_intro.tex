\section{Introduction}
\label{sec:intro}
%% ---------------------------------------------------------------------------

Steel manufacturing accounts for approximately 7--8\% of global CO$_2$
emissions~\cite{worldsteel2023}.
Electric arc furnaces (EAF) and basic oxygen furnaces (BOF) expel exhaust
streams that oscillate between 200\textdegree{}C and 1{,}200\textdegree{}C
with cycle periods of 1--15 minutes --- a transient heat source profile that
makes conventional thermodynamic bottoming cycles impractical and power
recovery control exceedingly difficult.
Recovering this thermal energy via a power cycle could displace significant
grid electricity; at 10~MWe per furnace cluster and typical steel mill
electricity prices, the economic potential is several million USD per year
per installation.
Sathish et al.~\cite{sathish2019sco2whr} demonstrated the thermodynamic
viability of sCO$_2$ Brayton cycles for steel plant waste heat recovery,
motivating the present work on autonomous control.

Supercritical CO$_2$ (sCO$_2$) Brayton cycles~\cite{dostal2004} are an
attractive bottoming cycle for this application: operating above the CO$_2$
critical point (31.1\textdegree{}C, 7.38~MPa) enables efficiencies of
27--40\% at compact turbomachinery scales that are
100$\times$ smaller than equivalent steam plant, making the technology
cost-competitive at industrial waste-heat magnitudes where steam plant
would be uneconomic.

However, the fluid's near-critical thermodynamic properties introduce severe
nonlinearity.
Specific heat peaks at $c_p \approx 29.6$~kJ\,kg$^{-1}$\,K$^{-1}$ near
35\textdegree{}C/80~bar --- more than $10\times$ the ideal-gas value.
A 1.5\textdegree{}C compressor inlet temperature drop demands 6\% more
cooling power, while the same temperature increase requires 18\% less:
a strongly asymmetric gain that defeats fixed-gain PID tuning during furnace
transients~\cite{sco2review2021}.
The compressor inlet must remain strictly above the CO$_2$ critical
temperature (31.1\textdegree{}C) to avoid two-phase flow that would damage
turbomachinery; this constraint becomes safety-critical during cold startup
and emergency trip scenarios.

Deep reinforcement learning (RL) offers an adaptive alternative: an agent
trained on a physics-faithful digital twin can learn to anticipate and exploit
thermodynamic nonlinearities without requiring an explicit analytical system
model.
Several key technical capabilities have matured that make this approach
practical.
First, the Functional Mockup Interface (FMI) standard enables physics-faithful
simulation at the speed required for RL training, and OpenModelica models of
sCO$_2$ cycles can be exported as FMI~2.0 Co-Simulation FMUs with embedded
stiff solvers.
Second, a lightweight residual MLP trained on $(s_t, a_t) \to s_{t+1}$
transitions enables GPU-vectorised RL at 250{,}000~steps/s --- over
$300\times$ faster than CPU FMU simulation --- while maintaining sub-1\%
state prediction error.
Third, Fourier Neural Operators~\cite{fno2021} learn physics operator mappings
with GPU-accelerated inference; NVIDIA PhysicsNeMo provides production-grade
GPU implementations achieving $R^2 = 1.000$ on held-out trajectories, although
the non-causal sequence-to-sequence architecture is incompatible with
step-by-step RL without architectural modifications (Section~\ref{sec:fno_surrogate}).
Fourth, Constrained Policy Optimisation variants~\cite{achiam2017cpo} with
trainable Lagrangian multipliers enforce operational constraints (compressor
inlet temperature, surge margin) throughout training and deployment.
Finally, ONNX export followed by TensorRT FP16 compilation achieves
sub-millisecond plant-edge inference latency, satisfying real-time control
requirements.

Related work has demonstrated RL on building energy systems via
Modelica/FMU environments~\cite{modelicagym,boptestgym}, and on organic
Rankine cycle superheat control for internal combustion engine
exhaust~\cite{wang2020orc}.
More recently, Zhu et al.~\cite{zhu2024fno} combined Fourier Neural Operators
with model predictive control for sCO$_2$ cycle dynamics.
The EU-funded iSOP doctoral network (Horizon Europe grant 101073266) trains
15 researchers specifically on sCO$_2$ transient modelling and novel control
strategies~\cite{isop2024}, underscoring community recognition of this
unsolved control problem.
Sathish et al.~\cite{sathish2024aif} have also proposed active inference as
an alternative advanced control paradigm for centrifugal compressor surge
control, demonstrating the broader trend towards intelligent, adaptive control
in turbomachinery applications.

Despite this growing interest, to the authors' knowledge no publicly available
framework combines (i)~a physics-faithful OpenModelica-exported sCO$_2$ FMU,
(ii)~structured curriculum RL with Lagrangian safety constraints,
(iii)~an MLP step-predictor surrogate enabling 250{,}000~steps/s GPU training,
(iv)~an NVIDIA PhysicsNeMo FNO surrogate path ($R^2 = 1.000$), and
(v)~a sub-millisecond TensorRT deployment artefact.
sCO2RL fills this gap with a fully open-source implementation targeting the
waste heat recovery application.

The contributions of this work are sixfold.
The first contribution is a publicly available Gymnasium environment wrapping
an OpenModelica-exported sCO$_2$ FMU with 14~observation variables,
4~actuator channels, a 7-phase structured curriculum, and Lagrangian safety
constraints enforcing operation above the CO$_2$ critical point.
Second, the paper demonstrates that PPO with Lagrangian constraints achieves
$+30$--$39\%$ cumulative episode reward improvement over Ziegler--Nichols-tuned
PID in steady-state and mild-transient scenarios (Phases~0--2), with zero
safety violations across 140~evaluation episodes.
Third, an MLP step-predictor surrogate (residual MLP, 4~layers, 512 hidden
units, trained on 55{,}000{,}000 $(s,a,s')$ transitions) enables
GPU-vectorised PPO at 250{,}000~steps/s, achieving 18.5$\times$ lower
net-power tracking error than the PID baseline in 23~minutes of training.
Fourth, the integration of NVIDIA PhysicsNeMo FNO surrogate training with
76{,}600 unique Latin Hypercube-sampled FMU trajectories ($R^2 = 1.000$)
is presented, along with empirical characterisation of both the data quality
failure mode causing surrogate fidelity collapse ($R^2 = -77$) under
dataset degeneracy and the FNO's architectural incompatibility with
step-by-step RL prediction.
Fifth, a TensorRT-FP16 deployment path achieving p99 inference latency
of 0.046~ms --- $22\times$ under the 1~ms plant-edge SLA --- is demonstrated.
Sixth, a detailed diagnosis of five non-algorithmic training infrastructure
defects encountered in practice is provided, covering observation normalisation
persistence, episode boundary detection, reward unit double-scaling, stale
disturbance profiles, and constraint-violation gating, as practitioner guidance
for the FMU-RL integration community.

The remainder of this paper is organised as follows.
Section~\ref{sec:related} reviews related work on sCO$_2$ cycle control,
reinforcement learning for thermodynamic systems, and FMU-based training
environments.
Section~\ref{sec:architecture} describes the system architecture, including
the physics simulation layer, Gymnasium environment, and surrogate models.
Section~\ref{sec:method} details the reward function, Lagrangian constraint
formulation, curriculum design, and PID baseline methodology.
Section~\ref{sec:results} presents experimental results for both the
FMU-direct and MLP surrogate training paths.
Section~\ref{sec:control_analysis} provides a control-theoretic analysis
comparing RL and PID controllers.
Section~\ref{sec:bugs} documents five practitioner-relevant engineering
defects, and Section~\ref{sec:conclusion} concludes with key findings and
future work.

%% ---------------------------------------------------------------------------
