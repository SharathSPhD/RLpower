\section{Introduction}
%% ---------------------------------------------------------------------------

Steel manufacturing accounts for approximately 7--8\% of global CO$_2$
emissions~\cite{worldsteel2023}.
Electric arc furnaces (EAF) and basic oxygen furnaces (BOF) expel exhaust
streams that oscillate between 200\textdegree{}C and 1{,}200\textdegree{}C
with cycle periods of 1--15 minutes.
Recovering this thermal energy via a power cycle could displace significant
grid electricity, but the extreme transients make control exceptionally
challenging.

Supercritical CO$_2$ power cycles~\cite{dostal2004} are an attractive
bottoming cycle for this application: operating above the CO$_2$ critical
point (31.1\textdegree{}C, 7.38~MPa) enables efficiencies of 27--40\% at
compact turbomachinery scale (10--100$\times$ smaller than steam equivalents)
that makes the technology cost-competitive at industrial waste heat magnitudes.
However, the fluid's near-critical thermodynamic properties introduce severe
nonlinearity: specific heat peaks at 29.6~kJ\,kg$^{-1}$\,K$^{-1}$ near
35\textdegree{}C/80~bar, so a 1.5\textdegree{}C compressor inlet temperature
drop demands 6\% more cooling power, while a 1.5\textdegree{}C increase
requires 18\% less — a strongly asymmetric gain that defeats fixed-gain PID
tuning during furnace transients~\cite{sco2review2021}.

Reinforcement learning (RL) offers an adaptive alternative: an agent trained
on a physics-faithful digital twin can learn to anticipate and exploit
thermodynamic nonlinearities without requiring an explicit system model.
Related work has demonstrated RL on building energy systems via
Modelica/FMU environments~\cite{modelicagym,boptestgym}, and on organic
Rankine cycle superheat control for internal combustion engine
exhaust~\cite{wang2020orc}.
More recently, Zhu et al.~\cite{zhu2024fno} combined Fourier Neural Operators
with reinforcement learning PI control for sCO$_2$ cycle dynamics, while a
comprehensive review by Liu et al.~\cite{sco2review2021} identifies advanced
data-driven control as a key research direction.
The EU-funded iSOP doctoral network (Horizon Europe grant 101073266) trains
15 researchers specifically on sCO$_2$ transient modelling and novel control
strategies~\cite{isop2024}, underscoring community recognition of this gap.

Despite this growing interest, to our knowledge no publicly available framework
combines (i)~a physics-faithful OpenModelica-exported sCO$_2$ FMU,
(ii)~structured curriculum RL with safety constraints,
and (iii)~a sub-millisecond deployment path.
sCO2RL fills this gap.

\textbf{Contributions.}
\begin{enumerate}
  \item A publicly available Gymnasium environment wrapping an
        OpenModelica-exported sCO$_2$ FMU with a 7-phase structured
        curriculum and Lagrangian safety constraints.
  \item PPO with trainable Lagrangian constraint multipliers for
        safe operation near the CO$_2$ critical region.
  \item An FNO surrogate path enabling GPU-vectorised training
        (${\approx}10^6$ steps/s vs.\ ${\approx}800$ steps/s on CPU FMU).
  \item A TensorRT-FP16 deployment artefact achieving p99 $<1$~ms.
  \item An honest, detailed diagnosis of five training infrastructure
        defects encountered in practice — including episode boundary
        misalignment, reward unit double-scaling, and normalisation
        persistence failure — as concrete practitioner guidance.
\end{enumerate}

%% ---------------------------------------------------------------------------
