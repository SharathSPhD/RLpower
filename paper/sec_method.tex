\section{Method}
\label{sec:method}
%% ---------------------------------------------------------------------------

This section details the algorithmic components of the sCO2RL framework:
the composite reward function, Lagrangian constraint formulation,
curriculum design, PID baseline methodology, and the Latin Hypercube
dataset collection procedure for surrogate training.

\subsection{Reward Function}

The reward decomposes into tracking, smoothness, and constraint terms:
\begin{align}
  r_t &= r_{\text{track}} + r_{\text{smooth}} + r_{\text{constraint}} \\
  r_{\text{track}} &= -|W_{\text{net,MW}} - W_{\text{demand}}|
                      \;\cdot\; w_{\text{track}} \\
  r_{\text{smooth}} &= -\|\Delta a_t\|^2 \;\cdot\; w_{\text{smooth}} \\
  r_{\text{constraint}} &= -\sum_i \lambda_i \cdot \mathbb{1}[\text{violation}_i]
\end{align}
where $W_{\text{net,MW}}$ is net shaft power in megawatts (converted from the
FMU's SI watts by the unit-conversion layer in \texttt{FMPyAdapter}),
$W_{\text{demand}}$ is the instantaneous demand setpoint set by the curriculum,
and $\lambda_i$ are the Lagrangian multipliers updated online.

A critical implementation detail concerns unit consistency: the FMU returns
power in watts, while the reward is designed around megawatts.
The unit conversion is applied at the \texttt{FMPyAdapter} level;
any additional scaling in the environment configuration must be set to $1.0$.
Failure to observe this leads to Bug~3 (Section~\ref{sec:bugs}): a
$10^{-6}$ double-scaling that collapses $r_{\text{track}}$ to machine-epsilon
magnitude, effectively reducing the reward signal to pure noise.

\subsection{Lagrangian Constraint Formulation}

Each constraint $c_i(s, a) \leq 0$ is enforced via a trainable multiplier
$\lambda_i \geq 0$:
\begin{equation}
  \mathcal{L}(\theta, \lambda) = J_r(\theta) - \sum_i \lambda_i \cdot J_{c_i}(\theta)
\end{equation}
where $J_r$ is the reward objective and $J_{c_i}$ is the expected constraint
cost over the policy $\pi_\theta$.
Policy parameters $\theta$ are updated via gradient ascent on $\mathcal{L}$;
multipliers $\lambda_i$ are updated via gradient ascent on $-\mathcal{L}$
(dual ascent), increasing the penalty when violations occur and decreasing it
when the constraint is comfortably satisfied.
This avoids trust-region constraint solves at each step while converging
to a Lagrangian saddle point in expectation.

The primary safety constraint is compressor inlet temperature:
\begin{equation}
  c_{\text{crit}}(s) = T_{\text{crit}} + 1\text{\textdegree{}C}
                       - T_{\text{comp,in}}(s) \leq 0
\end{equation}
with $T_{\text{crit}} = 31.1$\textdegree{}C, so the guard is
$T_{\text{comp,in}} \geq 32.1$\textdegree{}C.
Dropping below 31.5\textdegree{}C triggers hard episode termination with
reward $-100$ to prevent FMU solver divergence in the two-phase region.

\subsection{Curriculum Phases}
\label{sec:curriculum}

\begin{table}[h]
\centering
\caption{Seven-phase curriculum design.
         Episode lengths reflect the time horizon needed for each scenario
         to stabilise from a perturbed initial condition.
         Advancement requires mean episode reward above the threshold over
         a 50-episode rolling window, with constraint violation rate below 10\%.}
\label{tab:curriculum}
\begin{tabular}{clcrc}
\toprule
Phase & Scenario & Steps & Length & Advance threshold \\
\midrule
0 & Steady-state optimisation & 120 & 10 min & 8.0 \\
1 & $\pm30\%$ gradual load following & 360 & 30 min & 60.0 \\
2 & $\pm10$\textdegree{}C ambient disturbance & 720 & 60 min & 120.0 \\
3 & EAF heat source transients (200--1{,}200\textdegree{}C) & 1{,}080 & 90 min & 250.0 \\
4 & 50\% rapid load rejection ($<$30~s) & 360 & 30 min & 50.0 \\
5 & Cold startup through CO$_2$ critical region & 720 & 60 min & 80.0 \\
6 & Emergency turbine trip recovery & 360 & 30 min & 300.0 \\
\bottomrule
\end{tabular}
\end{table}

Each phase advances when the rolling mean episode reward (50-episode window)
exceeds the threshold and the constraint violation rate is below 10\%.
Phase advancement is checked every 10 episodes.
Regression to earlier phases is disabled to prevent oscillation; instead,
the agent accumulates training data at the current phase until it satisfies
the advancement condition.

The curriculum imposes progressively more extreme heat source variability
and control challenges: Phase~0 tests convergence to a fixed setpoint;
Phase~3 tests response to EAF-scale temperature transients with 1--15~minute
cycle periods; Phase~6 tests emergency response to sudden turbine isolation
with rapid inventory ejection under the Lagrangian constraint.

\subsection{PID Baseline: Ziegler--Nichols Tuning}
\label{sec:pid_tuning}

The PID baseline uses four independent parallel controllers, each mapping
one actuator to one process variable: the bypass valve controls turbine inlet
temperature, the inlet guide vane (IGV) angle controls main compressor inlet
temperature, the inventory valve position controls high-side pressure, and the
cooling flow fraction controls precooler outlet temperature.

Each controller implements a filtered derivative PID:
\begin{equation}
  u(t) = k_p \cdot e(t) + k_i \int_0^t e(\tau)\,d\tau
         + k_d \frac{d e^*}{dt}(t)
\end{equation}
where $e^*(t)$ is the filtered error signal through a first-order low-pass
filter with time constant $\tau_f$, preventing derivative kick on step inputs.

The Ziegler--Nichols tuning procedure is applied to each PID channel as follows.
A $10\%$ step input is applied to the corresponding actuator from the
nominal setpoint while holding all other actuators constant.
The step response is recorded and the process gain $K$, apparent dead time
$L$ (intercept of the inflection-point tangent), and time constant $T$
(tangent-axis crossing time minus $L$) are extracted.
The ZN gains are computed as $k_p = \frac{1.2 T}{K L}$,
$k_i = \frac{k_p}{2L}$, and $k_d = 0.5 k_p L$.
All gains are derated by $0.4\times$ to compensate for the ZN tendency to
produce oscillatory responses near stability limits in nonlinear systems.

The inventory valve (high-side pressure control) showed negligible step
response within the test window; manual gains were retained for that channel.
ZN-derived gains are stored in \texttt{artifacts/pid\_tuning/pid\_gains.json}
for reproducibility.

\subsection{LHS Dataset Collection for FNO Surrogate}
\label{sec:lhs_collection}

FNO surrogate training requires diverse state-space coverage to generalise
beyond the nominal operating point.
Latin Hypercube Sampling (LHS) generates $N$ samples from a 5-dimensional
initial-condition space (one dimension per actuator degree of freedom):
\begin{equation}
  \mathbf{x}_i^{(0)} \sim \text{LHS}(\mathbf{x}_{\min}, \mathbf{x}_{\max}, N=100{,}000)
\end{equation}
Each sample $\mathbf{x}_i^{(0)}$ initialises the FMU via \texttt{env.reset(options=\ldots)},
guaranteeing that each trajectory starts from a genuinely distinct operating point.

The LHS sampling scheme stratifies the $[0,1]^5$ unit hypercube into $N$
equal-probability strata (one per dimension), selects one sample per stratum,
and applies random shuffling to ensure uniformity without regularity artefacts.
This is strictly superior to random uniform sampling for small-$N$ regimes:
it guarantees no clustering and maximises coverage of the design space.

The initial implementation contained a bug in \texttt{SCO2FMUEnv.reset()}: the
\texttt{options} dictionary was accepted but not applied to the FMU initial
condition, so all 75{,}000 trajectories started from the same default operating
point.
Inspection revealed only 2{,}100 unique initial-state rows --- the effective
dataset size was 35$\times$ smaller than reported.
After diagnosing and fixing the \texttt{reset()} LHS application, a new
collection run was initiated targeting $100{,}000$ unique trajectories.
FMU solver instability in extreme LHS operating points (CVODE NaN propagation
in near-critical or high-turbine-inlet-temperature conditions) caused the
run to terminate at $76{,}600$ trajectories (76.6\% of target), yielding
3.98~GB of genuinely diverse training data.

%% ---------------------------------------------------------------------------
